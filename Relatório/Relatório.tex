\documentclass[12pt,a4paper]{instrumentacao}

\usepackage{array}
\usepackage{amsmath}
% desativado porque ler TNR é muito ruim! lembrar de desativar antes de imprimir
%\usepackage{times}

\title{Alguns dos Efeitos Físicos Explorados em Sensores}
\author{Rogiel Sulzbach \and Rodrigo de Castro Silveira \and Yi Chen Wu}
\startdate{28 de março de 2016}
\finishdate{04 de abril de 2016}
\emails{
	\emailaddress{R.J.S.}{rogiel@rogiel.com},
	\emailaddress{R.C.S.}{csilveira.rodrigo@gmail.com} e
	\emailaddress{Y.C.}{yichenpoa@gmail.com}
}
\resume{}
\abstract{}
\keywords{}
\institute{Universidade Federal do Rio Grande do Sul, Departamento de Engenharia Elétrica, Curso de Engenharia Elétrica, Instrumentação A, Profs. Dr. Alexandre Balbinot e Dra. Léia Bagesteiro}

\headertext{Efeitos físicos}

\begin{document}
\maketitle

\todo{mudar a letra para times new roman.......}

\chapter{Introdução}
Nesta atividade de laboratório exploramos os efeitos físicos de dois tipos de sensores. Um sensor potenciométrico e outro de efeito Hall linear.

\chapter{Metodologia Experimental}
\section{Sensor potênciométrico}
Para a atividade do sensor potenciométrico, desenvolvemos um pêndulo cujo eixo estava conectado à um potenciômetro. A construção está ilustrada na figura \ref{fig:estrutura-pendulo}:

\todo{figura}

\todo{equacionamento do pendulo}

Primeiramente, a fim de obter uma curva de calibração da estrutura, se fez necessário levantar estimativas do valor de resistência \todo{ou tensão} do potenciômetro em função do seu ângulo de oscilação. A fim de conseguir realizar o projeto em tempo hábil \todo{colocar isso mesmo?}, optamos por realizar as medidas em intervalos de 10 em 10 graus de inclinação no intervalo de -80º (oscilação ao lado esquerdo) e 80º (oscilação ao lado direito).

Com estes dados, e assumindo que o potenciômetro seja linear, podemos assumir que os valores intermediários de tensão mensuradas são proporcionais aos seus limites mensurados (valor medido superior e inferior).

\chapter{Resultados e Discussões}

\chapter{Conclusões}

\chapter*{Anexos}
\section{Mathematica}

\section{LabVIEW}

\section{MATLAB}

\begin{thebibliography}{9}
\bibitem{mathematica-numerial-precision} \url{https://reference.wolfram.com/language/tutorial/NumericalPrecision.html}, acessado em 16 de março de 2016
\bibitem{ref1} Sobrenome, A.B.; Sobrenome, C.D. Title of the cited article. Journal Title 2007, 6, 100-110. 
\bibitem{ref2} Balbinot, A.; Brusamarello, V.J.. Title of the cited article. Journal Title 2007, 6, 100-110. 
\bibitem{ref3} Author, A.; Author, B. Title of the chapter. In Book Title, 2nd ed.; Editor, A., Editor, B., Eds.; Publisher: Publisher Location, Country, 2007; Volume 3, pp. 154-196.
\bibitem{ref4} Author, A.; Author, B. Book Title, 3rd ed.; Publisher: Publisher Location, Country, 2008; 
pp. 154-196.
\todo{arrumar bibliografia}

\end{thebibliography}

\end{document}
