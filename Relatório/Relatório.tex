\documentclass[12pt,a4paper]{instrumentacao}

\graphicspath{
	{../Resources/Images/}
	{../Resources/Mathematica/images/}
}
\title{Alguns dos Efeitos Físicos Explorados em Sensores}
\author{Rogiel Sulzbach \and Rodrigo de Castro Silveira \and Yi Chen Wu}
\startdate{28 de março de 2016}
\finishdate{04 de abril de 2016}
\emails{
	\emailaddress{R.J.S.}{rogiel@rogiel.com},
	\emailaddress{R.C.S.}{csilveira.rodrigo@gmail.com} e
	\emailaddress{Y.C.}{yichenpoa@gmail.com}
}
\resume{}
\abstract{}
\keywords{}
\institute{Universidade Federal do Rio Grande do Sul, Departamento de Engenharia Elétrica, Curso de Engenharia Elétrica, Instrumentação A, Profs. Dr. Alexandre Balbinot e Dra. Léia Bagesteiro}

\headertext{Efeitos físicos}

\begin{document}
\maketitle

\todo{mudar a letra para times new roman.......}

\chapter{Introdução}
Nesta atividade de laboratório exploramos os efeitos físicos de dois tipos de sensores. Um sensor potenciométrico e outro de efeito Hall linear.

\chapter{Metodologia Experimental}
\section{Sensor potênciométrico}
Para a atividade do sensor potenciométrico, desenvolvemos um pêndulo cujo eixo estava conectado à um potenciômetro. A construção está ilustrada na figura \ref{fig:pendulo}:

\begin{figure}[H]
\centering
\includegraphics[width=0.4\textwidth]{Pendulo.pdf}
\caption{Estrutura do pêndulo}
\label{fig:pendulo}
\end{figure}

onde $\theta$ é o angulo de inclinação do pêndulo em relação a origem (linha tracejada).

Este pêndulo pode ser modelado eletricamente pelo seguinte circuito:

\begin{figure}[H]
\centering
\includegraphics[width=0.6\textwidth]{Pendulo-Circuito.pdf}
\caption{Modelo elétrico do pêndulo}
\label{fig:pendulo-equivalente}
\end{figure}

onde $R(angulo)$ é uma resistência variável em função do ângulo $\theta$.

A fim de extrair a função de transferência do sensor, realiza-se um experimento com repetição onde, através de um voltímetro, mede-se a tensão elétrica sobre o resistor $R(\theta)$ em função do angulo $\theta$. Com este experimento, pode-se realizar um \textit{fit} cuja função será utilizada como a função de transferência do sensor. Uma vez que o potenciometro é linear, espera-se que a função de transferência é dada na forma da equação \ref{eq:pendulo-tf-forma} e conforme a estrutura apresentada na figura \ref{fig:pendulo}:

\begin{equation}
	R(\theta) = R_0 + R_1 \times \theta
	\label{eq:pendulo-tf-forma}
\end{equation}

onde $R(\theta)$ é a resistência elétrica em do pêndulo $\Omega$ para um dado ângulo de inclinação, $R_0$ é a resistência do pêndulo em $\Omega$ no ângulo de 0 graus, $R_1$ é a variação de resistência elétrica do pêndulo em $\Omega / rad$ em função da variação de ângulo e $\theta$ é a inclinação do pêndulo em $rad$ relativa à sua posição de repouso.

Para minimizar o efeito de \textit{ripple} de tensão elétrica da fonte de alimentação, foram utilizados 2 reguladores de tensão elétrica um \textit{LM7805}\todo{colocar o fabricante, existem vários, confirmar em cima do CI}, regulador de $(5 \pm 0.2)V$\cite{datasheet-lm7805} e um regulador \textit{LM7905}\todo{colocar o fabricante, existem vários, confirmar em cima do CI} de $(-5 \pm 0.2)V$\cite{datasheet-lm7905}. O mesmo conjunto de reguladores foi utilizado para alimentar todos componentes externos.

\subsection{Aquisição da resposta temporal do pêndulo}

Para obter a resposta temporal do pêndulo, utilizamos um mecanismo de aquisição de dados conforme a figura \ref{fig:pendulo-fluxo-medidas}:

\begin{figure}[H]
\centering
\includegraphics[width=\textwidth]{Pendulo-Fluxograma.pdf}
\caption{Fluxo de aquisição de dados do sistema.}
$E_n$ indica as etapas de aquisição de dados.
\label{fig:pendulo-fluxo-medidas}
\end{figure}

\begin{itemize}
	\item \textbf{E1}: representa o sensor utilizado, no caso deste experimento um potenciometro;
	\item \textbf{E2}: representa um amplificador com ganho unitário (buffer);
	\item \textbf{E3}: representa o filtro anti-alias utilizado para o processo de amostragem;
	\item \textbf{E4}: representa o processo de amostragem realizado utilizando o LABView e o DAQ (Data Acquisition device)
	\item \textbf{E5}: no LABView foi armazenado os dados amostrados;
	\item \textbf{E6}: após o armazenamento dos dados foi realizado o processamento destes dados.
\end{itemize}

A cadeia de medidas deste experimento pode ser representada pelo diagrama de blocos da figura \ref{fig:pendulo-cadeia-medidas}:

\begin{figure}[H]
\centering
\includegraphics[width=\textwidth]{Cadeia de medida.pdf}
\caption{Cadeia de medida com grandezas de entrada e saída}
\label{fig:pendulo-cadeia-medidas}
\end{figure}

Para poder estimar o valor do angulo de inclinação do pêndulo em função da tensão elétrica medida pelo fluxo de medida mostrado na figura \ref{fig:pendulo-fluxo-medidas}, precisa-se definir a função de transferência para os 3 blocos definidos.

A função de transferência para o sensor já foi obtida no experimento anterior e é dada na equação \ref{eq:pendulo-tf-forma}. Contudo, para utilizá-la nas medidas realizadas, é preciso que a função de transferência inversa seja obtida, conforme a equação \ref{eq:pendulo-tf-forma-inversa}

\begin{equation}
	\theta = V^{-1}(\theta) = \theta_0 + \theta_1 \times V
	\label{eq:pendulo-tf-forma-inversa}
\end{equation}

onde $\theta$ é a inclinação do pêndulo em radianos, $V^{-1}(\theta)$ é a inversa da função de transferência do pêndulo modelada anteriormente, $\theta_0$ é o offset de angulo correspondente ao offset inicial de tensão elétrica e $\theta_1$ é a constante que define a variação de inclinação em função de uma variação de tensão elétrica do sensor.

A função de transferência do buffer de tensão é simples de modelar, uma vez que seu comportamento consiste em não realizar nenhuma alteração no valor de entrada, pode-se modelar usando uma função onde e saída é igual ao valor de tensão da entrada, conforme a equação \ref{eq:pendulo-tf-buffer}:

\begin{equation}
	V_{\text{buffer}} = 1 \times V_{\text{in}}	
	\label{eq:pendulo-tf-buffer}
\end{equation}

\todo{como modelar a TF do filtro?}

Por fim, pode-se calcular a função de transferência equivalente do sistema que é a concatenação de todas funções de transferência na sequência em que foram conectadas na cadeia de medida, a equação \ref{eq:pendulo-tf-concat} expressa esta relação de forma matemática:

\begin{equation}
	\theta = V_{\text{buffer}}(V_{\text{sensor}}(V_{\text{medido}}))
	\label{eq:pendulo-tf-concat}
\end{equation}

Substituindo-se estes valores, temos que a função de transferência completa do sistema é dada por \label{eq:pendulo-tf-bloco}

\begin{equation}
	\theta = V_{\text{sensor}}(V) = \theta_0 + \theta_1 \times V
	\label{eq:pendulo-tf-bloco}
\end{equation}

\chapter{Resultados e Discussões}

Com o pêndulo alimentado com a tensão elétrica simétrica de -5V e +5V, levantamos as seguintes medidas a fim de obter uma relação de tensão elétrica em função do angulo do pêndulo. A tabela \ref{tab:pendulo-calibracao} contém as medidas obtidas:

\begin{table}[H]
\centering
\caption{Tabela da relação entre temperatura e a resistência elétrica medida}
\begin{tabular}{|l|l|l|}
 \hline
 \textbf{Ângulo ($º$)} & \textbf{Tensão Elétrica ($V$)} & \textbf{Tensão Elétrica ($V$)} \\ \hline

 -60 & -1.886 & -1.886 	\\ \hline
 -55 & -1.702 & -1.702 	\\ \hline
 -50 & -1.564 & -1.564 	\\ \hline
 -45 & -1.426 & -1.426 	\\ \hline
 -40 & -1.196 & -1.196 	\\ \hline
 -35 & -1.127 & -1.127 	\\ \hline
 -30 & -0.92 & -0.92 	\\ \hline
 -25 & -0.736 & -0.736 	\\ \hline
 -20 & -0.621 & -0.621 	\\ \hline
 -15 & -0.46 & -0.46 	\\ \hline
 -10 & -0.345 & -0.345 	\\ \hline
 -5 & -0.184 & -0.184 	\\ \hline
 0 & 0. & 0. 			\\ \hline
 5 & 0.138 & 0.138 		\\ \hline
 10 & 0.322 & 0.322 	\\ \hline
 15 & 0.506 & 0.506 	\\ \hline
 20 & 0.69 & 0.69 		\\ \hline
 25 & 0.828 & 0.828 	\\ \hline
 30 & 0.966 & 0.966 	\\ \hline
 35 & 1.127 & 1.127 	\\ \hline
 40 & 1.265 & 1.265 	\\ \hline
 45 & 1.472 & 1.472 	\\ \hline
 50 & 1.587 & 1.587 	\\ \hline
 55 & 1.84 & 1.84 		\\ \hline
 60 & 1.955 & 1.955 	\\ \hline
 
\end{tabular}
\label{tab:pendulo-calibracao}
\end{table}
\todo{essa tabela ta totalmente errada. temos que refazer o experimento.} % eu fiz uma multiplicação mágica pra ajustar os valores que teríamos medido, mas precisamos mesmo fazer a medida certa dessa coisa, ou o Balbinot não vai aceitar

Observa-se que cada medida foi feita com duas repetições e de forma totalmente aleatória.

Podemos, adicionalmente, plotar estes pontos em um gráfico para melhor observar o formato da curva:

\begin{figure}[H]
\includegraphics[width=\textwidth]{Pendulo-Pontos.pdf}
\caption{Gráfico exibindo as medidas realizadas}
\label{fig:pendulo-medidas}
\end{figure}

É fácil observar que há uma tendência linear nas medidas, então podemos realizar um ajuste de curvas a fim de tentar obter a reta que melhor representa os dados medidos. Na equação \ref{eq:pendulo-model} temos o modelo de curva que fornece o melhor ajuste:

\begin{equation}
	V(\theta) = 0.0318532 \theta +0.02116
	\label{eq:pendulo-model}
\end{equation}

onde $V(\theta)$ é a tensão elétrica do pêndulo e $\theta$ é o angulo do pêndulo em graus.

Com esta equação também é possível calcular a sua função inversa e obter o ângulo em função da tensão elétrica, na equação \ref{eq:pendulo-model-inverse}, a função inversa do modelo é dada:

\begin{equation}
	\theta(V) = -0.664297 + 31.394 V
	\label{eq:pendulo-model-inverse}
\end{equation}

Dessa forma, podemos utilizar a equação \ref{eq:pendulo-model-inverse} para calcular o angulo de um pêndulo em função de seu correspondente valor de tensão elétrica conforme será feito a seguir.

A fim de obter uma curva de tensão elétrica em função do tempo, desenvolvemos uma aplicação no LABView que fosse capaz de exibir a curva extraída em tempo real na tela e pudesse exportar estes dados de forma que pudessem ser processados externamente. Na figura \ref{fig:pendulo-labview} está mostrado uma imagem do programa utilizado.

\todo{adicionar um print do labview}

Após realizar a exportação dos dados utilizamos o software Wolfram Mathematica para importar estes dados em formato TSV (tab-separated values) e realizar o processamento. Primeiramente, precisamos notar que a informação importada está em valores de tensão elétrica (Volt) e tempo (segundo). Primeiramente, desejamos converter estes valores de tensão elétrica em uma posição angular do pêndulo e para isto, utilizamos a função inversa calculada na equação \ref{eq:pendulo-model-inverse} e a aplicamos a cada ponto medido pelo DAQ pelo LABView. \todo{precisa usar o MATLAB aqui, arrumar o texto}

A figura \ref{fig:pendulo-curva-tensao-vs-tempo} apresenta a curva em forma bruta extraída diretamente do LABView:

\begin{figure}[H]
\centering
[criar a imagem]
\caption{Curva da tensão elétrica medida no pêndulo em função do tempo}
\label{fig:pendulo-curva-tensao-vs-tempo}
\end{figure}
\todo{curva do pendulo}

Sabendo que o ângulo de oscilação do pêndulo é dado pela equação \ref{eq:pendulo-model-inverse}, é possível aplicar esta equação a cada um dos valores de tensão elétrica amostrados e converter esta curva em uma curva de posição angular em função do tempo. O resultado desta transformação está na figura \ref{fig:pendulo-curva-angulo-vs-tempo}

\begin{figure}[H]
\centering
[criar a imagem]
\caption{Curva da posição angular do pêndulo em função do tempo}
\label{fig:pendulo-curva-tensao-vs-tempo}
\end{figure}
\todo{curva do pendulo}

\todo{sensibilidade, resolução de entrada, resolução de saída, erro de linearidade ou de conformidade, incerteza, etc, assim como, a correspondente Cadeia de Medidas Proposta (teórica) e a Cadeia de Medidas Experimental}

\todo{labview+matlab}
\todo{só matlab}

\todo{determinar a sensibilidade, a resolução de entrada, resolução de saída, o erro de linearidade (ou erro de conformidade se for o caso) de seu sistema}

\todo{Tarefa para Casa: de posse da função de medição de seu sistema determinar a incerteza combinada deste sistema.}

\todo{com os sinais armazenados pelo programa LabVIEW elaborar um procedimento para determinar o decaimento em função do tempo e discutir esse resultado.}

\todo{Medição da Amplitude.}
\todo{Medição de $\theta_0$}
\todo{incertezas dos itens anteriores e dos comprimentos}

\todo{amortecimento}

\todo{medição do periodo}
\todo{grafico de amplitude e periodo}

\todo{modelo de thevenin}
\todo{grafico saida do pot vs angulo. qual erro?}
\todo{sensibilidade do circuito anterior}

\todo{medida a 3 fios - só calcular}
\todo{correção pro offset do 3 fios: medida a 4 fios}

\todo{discutir quantidade de amostras (medidas obtidas) para cada ângulo versus tensão.}
\todo{calcular a incerteza de TODAS medidas}


\todo{experimento do efeito hall}
\todo{experimento do led}

\chapter{Conclusões}

\chapter*{Anexos}
\section{Mathematica}
%\includenotebook{../Resources/Mathematica/Experimento 1-1.pdf}{Experimento 1}

\section{LabVIEW}

\section{MATLAB}

\begin{thebibliography}{9}
\bibitem{mathematica-numerial-precision} \url{https://reference.wolfram.com/language/tutorial/NumericalPrecision.html}, acessado em 16 de março de 2016

%\bibitem{ref1} Sobrenome, A.B.; Sobrenome, C.D. Title of the cited article. Journal Title 2007, 6, 100-110. 
%\bibitem{ref2} Balbinot, A.; Brusamarello, V.J.. Title of the cited article. Journal Title 2007, 6, 100-110. 
%\bibitem{ref3} Author, A.; Author, B. Title of the chapter. In Book Title, 2nd ed.; Editor, A., Editor, B., Eds.; Publisher: Publisher Location, Country, 2007; Volume 3, pp. 154-196.
%\bibitem{ref4} Author, A.; Author, B. Book Title, 3rd ed.; Publisher: Publisher Location, Country, 2008; 
%pp. 154-196.

\todo{formatar essas referencias da forma correta... nao sei como ele quer}
\bibitem{datasheet-lm7805} Datasheet oferecido pelo fabricante do regulador de tensão LM7805
\bibitem{datasheet-lm7905} Datasheet oferecido pelo fabricante do regulador de tensão LM7905

\todo{arrumar bibliografia}

\end{thebibliography}

\end{document}
